\chapter{Summary and Conclusions}
\label{chap:final_summary}
\section{GPU Implementation}
Based on measured execution times of GPU kernels, multiple data-aided equalization filters where implemented for the purpose of equalizing an aeronautical telemetry channel.
Using GPU libraries and batch processing rather than custom designed GPU kernels produced massive speed ups.
Also, reformulating algorithms into frequency-domain convolution produced impressive speed ups.

For implementation in one Tesla K40c and two Tesla K20c GPUs the execution times for all equalizers met the real-time constraint.
It was shown that the frequency-domain equalizers are the easiest to implement and have the fastest execution time.
The CMA equalizer was shown to be the hardest to implement and has the slowest execution time.
The execution time did not provide the CMA the opportunity to iterate many times.
The ZF and MMSE equalizers were shown to be computationally challenging to implement but had an acceptable execution time.

Because data-aided equalizers are implemented for a real-time telemetry receiver system,
the execution time results must be considered along with the bit error rate performance.
%Despite the performance of the CMA equalizer in AWGN, the execution time per iteration does not allow the equalizer to converge in multipath.
As of this writing the FDE1 equalizer is recommended, marking the best tradeoff between performance and computational complexity.

\section{Contributions}
Through the years of working on the PAQ project I:
\begin{enumerate}
\item Implemented algorithms using batch GPU libraries to reduce execution time on average by 5.
\item Reduced GPU convolution time by using batch GPU librarys and cascading filters in the frequency domain.
\item Implemented linear solver libraries to make ZF and MMSE equalizers feasible and real-time.
\item Reformulated the CMA to leverage the speed of GPU convolution.
\item Implemented new ADC to drive success of the PAQ project.
\item Implemented resampling polyphase filters in GPUs.
\item Participated heavily in flight tests at Edwards AFB.
\item Presented a paper frequency offset compensation for equalized SOQPSK at International \newline Telemetering. Conference (ITC) \cite{ravert2016}.
\end{enumerate}

\section{Further Work}
The Levinson-Durbin algorithm GPU implementation only leveraged the toeplitz structure of the channel estimate auto-correlation matrix.
A hybrid sparse Levinson-Durbin algorithm could leverage the sparseness of the channel estimate auto-correlation matrix and the vector $\hat{\mathbf{h}}_{n_0}$.