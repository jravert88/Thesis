%%%%%%%%%%%%%%%%%%%%%%%%%%%%%%%%%%%%%%%%%%%%%%%%%%%%%%%%%%%%%%%%%%%%%%%%%%%%%%%%%%%%%%%%%%%%%%
%%%%%%%%%%%%%%%%%%%%%%%%%%%%%%%%%%%%%%%%%%%%%%%%%%%%%%%%%%%%%%%%%%%%%%%%%%%%%%%%%%%%%%%%%%%%%%
%%%%%%%%%%% GPU
%%%%%%%%%%%%%%%%%%%%%%%%%%%%%%%%%%%%%%%%%%%%%%%%%%%%%%%%%%%%%%%%%%%%%%%%%%%%%%%%%%%%%%%%%%%%%%
%%%%%%%%%%%%%%%%%%%%%%%%%%%%%%%%%%%%%%%%%%%%%%%%%%%%%%%%%%%%%%%%%%%%%%%%%%%%%%%%%%%%%%%%%%%%%%


% \cleardoublepage
\chapter{Equalizer Equations}
\label{sec:eq_eq}

\section{Overview}
There are 3 different kinds of equalizers I run
1. the solving ones!!!  They are equations like Ax=b where I have A and b but I need x
2. the initialized then iterative ones. CMA is initialized with MMSE then runs as many times a possible
3. the multiply ones! the FDEs are a simple multiply in the frequency domain

\section{Equations}

\subsection{The Solving Equalizers}


\subsubsection{The Zero-Forcing Equalizer}
The ZF equalizer was studied in the PAQ Phase 1 Final Report in ~equation 324
\begin{equation}
\mathbf{c}_\text{ZF} = (\mathbf{H}^\dagger \mathbf{H})^{-1} \mathbf{H}^\dagger \mathbf{u}_{n_0}
\label{eq:c_ZF_pinv}
\end{equation}
where $\mathbf{c}_\text{ZF}$ is a $L_{eq} \times 1$ vector of equalizer coefficients computed to invert the channel estimate $\mathbf{h}$.
The channel estimate is used to build the $L_{eq}+N_1+N_2 \times L_{eq}$ convolution matrix
\begin{equation}
\mathbf{H} = 
		\begin{bmatrix}
		h(-N_1)		&  			& 		 	&  			\\
		h(-N_1+1) 	& h(-N_1)	& 		 	&  			\\
		\vdots	 	& \vdots	& \ddots 	&  			\\
		h(N_2)		& h(N_2-1) 	&  			& h(-N_1)  	\\
		 			& h(N_2) 	&  			& h(-N_1+1) \\
		 			&  	   		&  			& \vdots	\\
		 			&  	   		&  			& h(N_2)	\\
	\end{bmatrix}.
\end{equation}
and $\mathbf{u}_{n_0}$ is the desired channel impulse response centered on $n0 = N_1+L_1+1$
\begin{equation}
\mathbf{u}_{n_0} = \begin{bmatrix} 0 \\ \vdots \\ 0 \\ 1 \\ 0 \\ \vdots \\ 0 \end{bmatrix}
	\begin{matrix*}[l] \left. \vphantom{\begin{matrix} 0 \\ \vdots \\ 0 \end{matrix}} \right\}
		\text{$n_0-1$ zeros}
		\\ \\
		\left. \vphantom{\begin{matrix} 0 \\ \vdots \\ 0 \end{matrix}} \right\}
		\text{$N_1+N_2+L_1+L_2-n_0+1$ zeros}
		\end{matrix*}.
\end{equation}
The computation of the coefficients in Equation \eqref{eq:c_ZF_pinv} can be simplified in a couple of ways: First the matrix multiplication of $\mathbf{H}^\dagger$ and $\mathbf{H}$ is the autocorrelation matrix of the channel
\begin{equation}
\mathbf{R}_{h} = 
\mathbf{H}^\dagger \mathbf{H} = 
		\begin{bmatrix}
		r_{h}(0)		& r^\ast_{h}(1)	& \cdots 	& r^\ast_{h}(L_{eq}-1)  	\\
		r_{h}(1) 		& r_{h}(0)		& \cdots 	& r^\ast_{h}(L_{eq}-2)  	\\
		\vdots	 			& \vdots				& \ddots 	&  							\\
		r_{h}(L_{eq}-1)	& r_{h}(L_{eq}-2)	& \cdots	& r_{h}(0)  			
	\end{bmatrix}
	\label{eq:R_h}
\end{equation}
where
\begin{equation}
r_{h}(k) = \sum_{n=-N_1}^{N_2} h(n) h^\ast(n-k).
\end{equation}
Second the matrix vector multiplication of $\mathbf{H}^\dagger$ and $\mathbf{u}_{n_0}$ is simply the $n_0$th row of $\mathbf{H}^\dagger$ or the conjugated $n_0$th column of $\mathbf{H}$.
A new vector $\mathbf{h}_{n_0}$ is defined by
\begin{equation}
\mathbf{h}_{n_0} = \mathbf{H}^\dagger \mathbf{u}_{n_0} = 
\begin{bmatrix} h(L_1) \\ \vdots \\ h(0) \\ \vdots \\ h(-L_2)  \end{bmatrix}.
\label{eq:h_no}
\end{equation}
Replacing the matrix multiplication $\mathbf{H}^\dagger \mathbf{H}$ and $\mathbf{H}^\dagger \mathbf{u}_{n_0}$ simplifies Equation \eqref{eq:c_ZF_pinv} to
\begin{equation}
\mathbf{c}_\text{ZF} = \mathbf{R}^{-1}_{h} \mathbf{h}_{n_0}.
\label{eq:c_ZF_R_h}
\end{equation}

Computing the inverse of $\mathbf{R}_{h}$ is computationally heavy because an inverse is an $N^3$ operation.
To avoid an inverse, $\mathbf{R}_{h}$ is moved to the left side and $\mathbf{c}_\text{ZF}$ is found by solving a system of linear equations. 
Note that $r_{h}(k)$ only has support on $-L_{ch} \leq k \leq L_{ch}$ making $\mathbf{R}_{h}$ sparse or $\%63$ zeros.
The sparseness of $\mathbf{R}_{h}$ can be leveraged to reduce computation drastically.
The Zero-Forcing Equalizer coefficients are computed by solving
\begin{equation}
\mathbf{R}_h \mathbf{c}_\text{ZF} = \mathbf{h}_{n_0}.
\label{eq:c_ZF_solve}
\end{equation}


\subsubsection{MMSE Equalizer}
The MMSE equalizer was studied in the PAQ Phase 1 Final Report in ~equation 330.
\begin{equation}
\mathbf{c}_{MMSE} = \big[ \mathbf{G}\mathbf{G}^\dagger + \frac{\sigma_w^2}{\sigma_s^2}\mathbf{I}_{L_1+L_2+1} \big] \mathbf{g}^\dagger
\label{eq:MMSE_start}
\end{equation}
where
\begin{equation}
\mathbf{G} = 
		\begin{bmatrix}
		h(N_2)		& \cdots	& h(-N_1) 	&  			\\
					& h(N_2)	& \cdots 	& h(-N_1)	\\
				 	& 			& \ddots 	&  			& \ddots	\\
		 			&  	   		&  			& h(N_2)	& \cdots	& h(-N_1)	\\
	\end{bmatrix}
\end{equation}
and
\begin{equation}
\mathbf{g} = \begin{bmatrix} h(L_1) \cdots h(-L_2) \end{bmatrix}.
\end{equation}
The matrix multiplication $\mathbf{G}\mathbf{G}^\dagger$ is the same autocorrelation matrix $\mathbf{R}_h$ as Equation \eqref{eq:R_h}.
The vector $\mathbf{g}^\dagger$ is also the same vector as $\mathbf{h}_{n_0}$.
The signal-to-noise ratio estimate $\frac{1}{2\hat{\sigma^2_w}}$ is substituted in for the fraction $\frac{\sigma_w^2}{\sigma_s^2}$ using Equation 333 Rice's report.
Equation \eqref{eq:MMSE_start} can be reformulated to
\begin{equation}
\big[ \mathbf{R}_h + \frac{1}{2\hat{\sigma^2_w}} \mathbf{I}_{L_1+L_2+1} \big] \mathbf{c}_\text{MMSE} = \mathbf{h}_{n_0}.
\label{eq:c_MMSE_solve1}
\end{equation}
The MMSE Equalizer coefficients are solved for in a similar fashion to Zero-Forcing is in Equation \eqref{eq:c_ZF_solve}.
The only difference between Equation \eqref{eq:c_MMSE_solve1} and \eqref{eq:c_ZF_solve} is the noise variance is added down the diagonal of $\mathbf{R}_h$.
The matrix $\mathbf{R}_{hw}$ is defined to make the computation of the MMSE Equalizer coefficients the same as the Zero-Forcing Equalizer coefficients by adding the noise variance to the diagonal of $\mathbf{R}_h$
\begin{equation}
\mathbf{R}_{hw} = 
\mathbf{H}^\dagger \mathbf{H} = 
		\begin{bmatrix}
		r_{h}(0) + \frac{1}{2\hat{\sigma^2_w}}	& r^\ast_{h}(1)							& \cdots 	& r^\ast_{h}(L_{eq}-1)  	\\
		r_{h}(1) 								& r_{h}(0) + \frac{1}{2\hat{\sigma^2_w}}& \cdots 	& r^\ast_{h}(L_{eq}-2)  	\\
		\vdots	 								& \vdots								& \ddots 	&  							\\
		r_{h}(L_{eq}-1)							& r_{h}(L_{eq}-2)						& \cdots	& r_{h}(0) + \frac{1}{2\hat{\sigma^2_w}}  			
	\end{bmatrix}.
	\label{eq:R_MMSE}
\end{equation}
The MMSE Equalizer coefficients are computed by solving
\begin{equation}
\mathbf{R}_{hw}\mathbf{c}_\text{MMSE} = \mathbf{h}_{n_0}.
\label{eq:c_MMSE_solve}
\end{equation}
\clearpage
\subsection{The Iterative Equalizer}

\subsubsection{The Constant Modulus Algorithm}
CMA uses a steepest decent algorithm.
\begin{equation}
\mathbf{c}_{b+1} = \mathbf{c}_{b}-\mu \nabla \mathbf{J}
\end{equation}
The vector $\mathbf{J}$ is the cost function and $\nabla J$ is the cost function gradient defined in the PAQ report 352 by
\begin{equation}
	\nabla J = \frac{2}{L_{pkt}} \sum_{n=0}^{L_{pkt}-1}
	\left[ \vphantom{\displaystyle\sum}  y(n) y^\ast(n) - R_2 \right]
	y(n)  \mathbf{r}^\ast(n).
\label{eq:DelJcma-approxr}
\end{equation}
where
\begin{equation}
\mathbf{r}(n) = \begin{bmatrix} r(n+L_1) \\ \vdots \\ r(n) \\ \vdots \\ r(n-L_2) \end{bmatrix}.
\end{equation}
This means $\nabla J$ is of the form
\begin{equation}
\nabla J = \begin{bmatrix} \nabla J(-L_1) \\ \vdots \\ \nabla J(0) \\ \vdots \\ \nabla J(L_2) \end{bmatrix}.
\end{equation}
To Leverage computational efficiency of FFT, re-express the elements of $\nabla J$ as a convolution.

To begin define
\begin{equation}
z(n) = 	2\left[ \vphantom{\displaystyle\sum}  y(n) y^\ast(n) - R_2 \right] y(n)
\end{equation} 
so that $\nabla J$ may be expressed as
\begin{multline}
\nabla J
	= 
	\frac{z(0)}{L_{pkt}}
		\begin{bmatrix} r^\ast(L_1) \\ \vdots \\ r^\ast(0) \\ \vdots \\ r^\ast(L_2) \end{bmatrix} +
	\frac{z(1)}{L_{pkt}}
		\begin{bmatrix} r^\ast(1+L_1) \\ \vdots \\ r^\ast(1) \\ \vdots \\ r^\ast(1-L_2) \end{bmatrix} + \cdots
	\frac{z(L_{pkt}-1)}{L_{pkt}}
		\begin{bmatrix} r^\ast(L_{pkt}-1+L_1) \\ \vdots \\ r^\ast(L_{pkt}-1) \\ \vdots \\ r^\ast(L_{pkt}-1-L_2) \end{bmatrix}
\label{eq:delJ_writeoutr}.
\end{multline}
The $k$th value of $\nabla J$ is
\begin{equation}
\nabla J(k) = \frac{1}{L_{pkt}} \sum^{L_{pkt}-1}_{m=0}  z(m) r^\ast(m-k), \quad -L_1 \leq k \leq L_2.
\end{equation}
The summation almost looks like a convolution.
To put the summation in convolution form, define
\begin{equation}
\rho(n) = r^\ast(n).
\end{equation}
Now
\begin{equation}
\nabla J(k) = \frac{1}{L_{pkt}} \sum^{L_{pkt}-1}_{m=0}  z(m) \rho(k-m).
\label{eq:CMA_delJ_rice_reformed}
\end{equation}
Because $z(n)$ has support on $0 \leq n \leq L_{pkt}-1$ and $\rho(n)$ has support on $-L_{pkt}+1 \leq n \leq 0$, 
the result of the convolution sum $b(n)$ has support on $-L_{pkt}+1 \leq n \leq L_{pkt}-1$.
Putting all the pieces together, we have
\begin{align}
b(n) &= \sum^{L_{pkt}-1}_{m=0} z(m) \rho(n-m) \nonumber \\
	 &= \sum^{L_{pkt}-1}_{m=0} z(mr^\ast(m-n)
	 \label{eq:CMA_conv_z_rho}
\end{align}
Comparing Equation \eqref{eq:CMA_delJ_rice_reformed} and \eqref{eq:CMA_conv_z_rho} shows that 
\begin{equation}
\nabla J(k) = \frac{1}{L_{pkt}} b(k), \quad -L_1 \leq k \leq L_2.
\label{eq:CMA_delJ_donzo}
\end{equation}
The values of interest are shown in Figure Foo!!!!(c)

This suggest the following algorithm for computing the gradient vector $\nabla J$:
Matlab Code!!!